
\documentclass[oneside,master]{BITthesis}

%% openany,|openright,可选,双面打印时每章的第一页仅右页开启,默认右页开启(openright)
%% oneside|twoside ,可选,默认开启oneside
%% master|doctor,必需,博士采用doctor选项,默认选项master为硕士

%%-----------可以加载自己需要的包,绝大部分包在源程序中已经添加-------------

\usepackage{dirtree}
\newcommand\myicon[1]{{\color{#1}\rule{2ex}{2ex}}}
\newcommand{\myfolder}[2]{\myicon{#1}\ {#2}}

%%-----------以下是封面的填入栏目,按照格式填入即可================

\zhtitle{BITthesis 研究生论文模板说明和一些使用技巧}
\entitle{BITthesis manual and some tips about \LaTeXe{}}
\zhauthor{北理联盟}
\enauthor{lotus}
\zhadvisor{***}
\enadvisor{David}
\zhchairman{哈教授}
\enchairman{Dr.Wanfg}
\zhdegree{工程硕士}
\endegree{Master of Engineering}
\zhdate{2017年7月7号}
\endate{2017}
\zhacademy{宇航学院}
\enacademy{Aerospace Engineering}
\zhmajor{航空宇航科学与技术}
\enmajor{Aerospce tecnology}
\zhschool{北京理工大学}
\enschool{Beijing Institute of Technology}
\classified{TQ028.1}
\udc{540}


\begin{document}
	
\makecover     %%% 封面部分

\begin{zhabstract}
 临近毕设,学院里提供了毕业设计的 \LaTeX{}模板意见征求稿,于是决定离开简单的 Markdown ,开始使用 \LaTeX{} 进行写作。但是发现研究生院提供的模板格式存在很大的问题,编译也出现错误,于是修改重新设计了模板,欢迎再欢迎提交建议和意见,欢迎高质量的PR。项目地址为\url{https://github.com/qiuzhu/BITthesis}
 
\end{zhabstract}

\zhkeywords{模板;北京理工大学;毕业设计}

\begin{enabstract}
 Fluid throat nozzle technology is a new thrust vector control technology, which is generally used in solid rocket motors used in mechanical thrust vector control technology has the advantage of not driving the drive, such as Hou bolt, etc., reducing the Structure size and quality, and no moving parts, making the reliability enhancement. Fluid throat nozzle technology is generally through the secondary fluid injection, the mainstream and secondary flow interaction, making the mainstream of the throat flow area and throat shape changes.
 
 
\end{enabstract}

\enkeywords{template; \LaTeX; Github; experiment; BIT}

%% 符号对照表,可选,如不用可注释掉
\begin{denotation}

\item[HPC] 高性能计算 (High Performance Computing)
\item[cluster] 集群
\item[Itanium] 安腾
\item[SMP] 对称多处理
\item[API] 应用程序编程接口
\item[PI]	聚酰亚胺
\item[MPI]	聚酰亚胺模型化合物,N-苯基邻苯酰亚胺
\item[PBI]	聚苯并咪唑
\item[MPBI]	聚苯并咪唑模型化合物,N-苯基苯并咪唑
\item[PY]	聚吡咙
\item[PMDA-BDA]	均苯四酸二酐与联苯四胺合成的聚吡咙薄膜
\item[$\Delta G$]  	活化自由能~(Activation Free Energy)
\item [$\chi$] 传输系数~(Transmission Coefficient)
\item[$E$] 能量
\item[$m$] 质量
\item[$c$] 光速
\item[$P$] 概率
\item[$T$] 时间
\item[$v$] 速度
\item[劝  学] 君子曰:学不可以已。青,取之于蓝,而青于蓝;冰,水为之,而寒于水。
  木直中绳。(车柔)以为轮,其曲中规。虽有槁暴,不复挺者,(车柔)使之然也。故木
  受绳则直, 金就砺则利,君子博学而日参省乎己,则知明而行无过矣。吾尝终日而思
  矣,  不如须臾之所学也;吾尝(足齐)而望矣,不如登高之博见也。登高而招,臂非加
  长也,  而见者远;  顺风而呼,  声非加疾也,而闻者彰。假舆马者,非利足也,而致
  千里;假舟楫者,非能水也,而绝江河,  君子生非异也,善假于物也。积土成山,风雨
  兴焉;积水成渊,蛟龙生焉;积善成德,而神明自得,圣心备焉。故不积跬步,无以至千
  里;不积小流,无以成江海。骐骥一跃,不能十步;驽马十驾,功在不舍。锲而舍之,朽
  木不折;  锲而不舍,金石可镂。蚓无爪牙之利,筋骨之强,上食埃土,下饮黄泉,用心
  一也。蟹六跪而二螯,非蛇鳝之穴无可寄托者,用心躁也。
\end{denotation}

\tableofcontents
%% 插图索引,可选,如不用可注释掉
\listoffigures

\mainmatter  %%% 主体部分(绪论开始,结论为止)
%%======================================================================%%
%* 子文件的多少和内容由你决定(最好以章为单位),基本原则是提速预览、脉络清晰、管理容易。
%%======================================================================%%


\chapter{\LaTeX{}介绍}
\section{使用\LaTeX{}的写毕业论文的有点}
有人还写过论文,参见\url{http://journals.plos.org/plosone/article?id=10.1371/journal.pone.0115069}
在我看来,最大的优点在于:
\begin{itemize}
	\item 数学公式的自动编号和交叉引用\cite{merry_1975_PenetrationVerticalJets}
	\item 文件干净,随手记事本或者Vim或者nano都能编辑,不像Word的docx解压以后一堆人眼无法阅读的xml文档\cite{mendez_1998_StudyGasVelocity}
	\item 因为文件干净,自动化也很方便,Bash、Python……都可以干活(当然Word也可以通过VBS和C进行很强大的自动化)
	\item 强迫用户以结构化的方式写作,输出的PDF结构树清晰\cite{crawford_1995_FutureLibrariesDreams}
	而Word默认导出PDF是不输出结构的,需要另外勾选,当然如果勾选了的话不比LaTeX差[附图1]
	\item 各种各样的宏包,TikZ这种包估计Word万年都不会有对应的插件\cite{allen_1996_ReviewPerformanceEngineering,bilitza_2014_InternationalReferenceIonosphere,bilitza_2012_MeasurementsIRIModel,bilitza_2008_InternationalReferenceIonosphere,zhanghui_2012_WoGuoCuiHuaLieHuaGongYiJiZhuFaZhanYuQuShi}
	\item 模板质量都很高,各种边距都考虑得很周到,而且切换方便,可以管理的格式很多,如[1]中提到的分栏问题Word的模板是解决不了的,因为本质上Word里“分栏”是页面的属性而不是段落的属性UNIX-friendly
	\item 长度单位不依赖于系统的地区设置各种特殊页面界定清晰,修改灵活,不像Word的“封面”功能有些莫名奇妙~
	\item 矢量图只要用了合适的包和编译引擎就能支持很多格式,不像Word只支持emf或者wmf题注系统比Word强到不知道哪里去了
    \item Computer Modern系列字体是真的美,美出声。
\end{itemize}


\section{公式举例}
附录有跟多类型的公式,可以观摩此公式编辑功能的美:

线性稳定性主要研究小扰动振幅的变化规律,主要基础为燃烧室内的一维波动方程。燃烧室内压力$P$可表示为:
\begin{equation}
P=\dot{p}+p_0 e^{\alpha t} e^{j(\omega t+hx)}\label{eq:pressue p}
\end{equation}
若$\alpha>0$,小扰动有增长趋势,则燃烧不稳定;若$\alpha>0$,则小扰动有减弱趋势,燃烧具有稳定性,其增长常数α可表示为各种增益、阻尼效果之和,如式\ref{eq:zengyi}所示。
\begin{equation}
\alpha=\alpha_{pc}+\alpha_{vc}+\alpha_{dc}+\alpha_n+\alpha_p+\alpha_{mf}+\alpha_{g}+\alpha_{w}+\alpha_{st}\label{eq:zengyi}
\end{equation}

\section{一些题外话}
\subsection{内容为王}
论文当然是内容为王,不应该是被格式分心
\begin{itemize}
	\item 不会用LaTeX --> 无法编译 没有文档
\item	不会用word --> 文档真难看 格式丑死了
	
\item 	会用LaTeX --> 漂亮的文档
\item	会用word --> 文档
	
\item	LaTeX 用的好 --> 牛逼的文档
\item	Word 用的好 --> 牛逼的文档
\end{itemize}

\subsection{word VS \LaTeX{}}
能用LaTeX的人,通常知道如何正确地使用LaTeX;能用Word的人,大多数根本就不会正确地使用Word,比如样式模板、“内容和样式分开管理”、域代码、VBA……而且上面好多人说的LaTeX可以直接套现成的模板……那是模板的功劳,幸好本文编写好了北理工的研究生模板啦!

总之:
\begin{itemize}
	\item word是开始觉得容易,后来觉得难,并且发现越来越难
	\item latex是开始觉得难,后来觉得容易,往后又发现难而且非常难,所以就凑合着用了,好在模板很多
\end{itemize}

\section{本文主要研究内容}
关于此模板的使用


  %%第一章内容
\chapter{如何使用BITthesis模板}
\section{样例项目}
我对研究生院提供的LaTeX 模板进行了较多的修改,使其符合了学院方面的最新格式要求,并且修复了设置字号时行间距不正确等等错误,可以直接使用或者用于参考学习:

\section{稳态波衰减法和脉冲衰减法}
目前已经发展了多种冷流模拟试验,有直接法[198]、稳态波衰减法[199]、频率响应法[200]、脉冲法[7]和阻抗管法[201]。冷气状态与真实发动机工作状态之间的区别是介质温度和成分不同,因而平均声速及特征声阻抗也不同,因此,需要将冷态试验数据做进一步转换才能去表征真实发动机的喷管阻尼系数。通过试验能够测量不同复杂结构的喷管阻尼特性,但是试验方法成本较高而且试验周期较长。近年来,研究人员通过数值方法对喷管阻尼特性进行了计算分析[202][203][204],数值计算结果能够很好地吻合试验结果,为喷管阻尼研究提供了便利的数值工具。
本节将重点介绍稳态波衰减法及脉冲衰减法测量喷管阻尼常数的试验原理,通过借鉴试验原理,提出相应的数值计算方法。采用稳态波衰减法测量喷管阻尼时,首先需要调节声源,使其频率等于模拟燃烧室中的某阶固有频率,待燃烧内建立稳定的驻波以后,突然切断声源,当声源切断后,燃烧室内的压力将以指数形式衰减,通过动态压力监测系统记录测量点的瞬态压力曲线,计算压力衰减系数,计算所得衰减系数即为喷管阻尼,

稳态波衰减法原理是燃烧室空腔内施加一个稳定的、周期性的正弦压力波动信号,待发动机内建立起稳定的波形以后,突然切断稳定的压力波动源,让燃烧室内压力自动衰减,通过记录不同点的瞬态压力曲线,即可得到压力衰减速率。
压力振荡幅值与初始压强幅值之间的关系如下式所示:
\begin{equation}
P=p_0 e^{\alpha t}\label{eq:yaqiang}
\end{equation}
其中,为发动机内的衰减系数。在本研究中,可视为壁面阻尼,气体粘性损失,喷管阻尼等因素的总合,是发动机内的整体阻尼。需要说明的是,由于数值计算仅仅为纯流动仿真,最后所得的衰减系数与理论值会有较大的区别,但是,该方法可用来横向比较不同结构发动机内的阻尼特性,为优化发动机设计提供一定的工程指导。

可利用下式计算阻尼值:
\begin{equation}
\alpha=\dfrac{lnp_2-lnp_1}{t_2-t_1}\label{eq:zuning}
\end{equation}
  %%第二章内容
\begin{summary}
本项目开源托管于Github,欢迎提交建议和意见,欢迎高质量的PR。项目地址为\url{https://github.com/qiuzhu/BITthesis}
由于时间非常仓促,这个模板肯定存在不少问题,所以我需要大家帮助一起解决这
些问题。
下面是我自认为模板需要改进的地方。
\begin{itemize}
	\item 页边距设置上好像有问题,不同打印机打印效果不同;
	\item 我只在ubuntu TEXLIVE2016 以及最近版的MIktex套装上做过测试,理论上任何其他环境是可行;
	\item 本模板仅仅使用与硕士与博士、不适合博士
\end{itemize}

大家的反馈为模板提高带来机会。
Happy Texing!
祝你好运!

\bigskip
\hfill ——汪卫 2016年12月7日

\end{summary}
 %% 全文总结


%%==============参考文献===================%%
%%==== 使用 BibTeX 编译========
% 注意:至少需要引用一篇参考文献,否则下面两行会引起编译错误
\nocite{*} %此处为测试,可以屏蔽掉
\bibliographystyle{gbt-7714-2015-numerical}  %此处或者更新2005.bst样式
\bibliography{reference/test,reference/contents}

%%============附录部分===========

\begin{appendix}
\addcontentsline{toc}{chapter}{附录}
\renewcommand\theequation{\Alph{chapter}--\arabic{equation}}  % 附录中编号形式是"A-1"的样子
\renewcommand\thefigure{\Alph{chapter}--\arabic{figure}}
\renewcommand\thetable{\Alph{chapter}--\arabic{table}}

\chapter{模板更新记录}
\label{chap:updatelog}

\textbf{2012年12月} 根据上海交通大学的模板由大眼睛,按照BIT的规范,构建了此模板。
\textbf{2017年6月} 根据最新研究生院要求重新设计,构建此模板
 % 此处为附录A,自己添加内容
%% app2.tex for SJTU Master Thesis
%% based on CASthesis
%% modified by wei.jianwen@gmail.com
%% version: 0.3a
%% Encoding: UTF-8
%% last update: Dec 5th, 2010
%%==================================================

\chapter{Maxwell Equations}


因为在柱坐标系下,$\overline{\overline\mu}$是对角的,所以Maxwell方程组中电场$\bf
E$的旋度

所以$\bf H$的各个分量可以写为:
\begin{subequations}
  \begin{eqnarray}
    H_r=\frac{1}{\mathbf{i}\omega\mu_r}\frac{1}{r}\frac{\partial
      E_z}{\partial\theta } \\
    H_\theta=-\frac{1}{\mathbf{i}\omega\mu_\theta}\frac{\partial E_z}{\partial r}
  \end{eqnarray}
\end{subequations}
同样地,在柱坐标系下,$\overline{\overline\epsilon}$是对角的,所以Maxwell方程组中磁场$\bf
H$的旋度
\begin{subequations}
  \begin{eqnarray}
    &&\nabla\times{\bf H}=-\mathbf{i}\omega{\bf D}\\
    &&\left[\frac{1}{r}\frac{\partial}{\partial
        r}(rH_\theta)-\frac{1}{r}\frac{\partial
        H_r}{\partial\theta}\right]{\hat{\bf
        z}}=-\mathbf{i}\omega{\overline{\overline\epsilon}}{\bf
      E}=-\mathbf{i}\omega\epsilon_zE_z{\hat{\bf z}} \\
    &&\frac{1}{r}\frac{\partial}{\partial
      r}(rH_\theta)-\frac{1}{r}\frac{\partial
      H_r}{\partial\theta}=-\mathbf{i}\omega\epsilon_zE_z
  \end{eqnarray}
\end{subequations}
由此我们可以得到关于$E_z$的波函数方程:
\begin{eqnarray}
  \frac{1}{\mu_\theta\epsilon_z}\frac{1}{r}\frac{\partial}{\partial r}
  \left(r\frac{\partial E_z}{\partial r}\right)+
  \frac{1}{\mu_r\epsilon_z}\frac{1}{r^2}\frac{\partial^2E_z}{\partial\theta^2}
  +\omega^2 E_z=0
\end{eqnarray}
 % 此处为附录A,自己添加内容
\end{appendix}

% 发表文章目录
\backmatter
%%==================================================
%% pub.tex for SJTU Master Thesis
%% based on CASthesis
%% modified by wei.jianwen@gmail.com
%% version: 0.3a
%% Encoding: UTF-8
%% last update: Dec 5th, 2010
%%==================================================

\begin{publications}{99}

    \item\textsc{Chen H, Chan C~T}. {Acoustic cloaking in three dimensions using acoustic metamaterials}[J].
      Applied Physics Letters, 2007, 91:183518.

    \item\textsc{Chen H, Wu B~I, Zhang B}, et al. {Electromagnetic Wave Interactions with a Metamaterial Cloak}[J].
      Physical Review Letters, 2007, 99(6):63903.
    
\end{publications}


% 致谢
%%==================================================
%% thanks.tex for SJTU Master Thesis
%% based on CASthesis
%% modified by wei.jianwen@gmail.com
%% version: 0.3a
%% Encoding: UTF-8
%% last update: Dec 5th, 2010
%%==================================================

\begin{thanks}

  {\color{red} 希望有牛人来维护咱们学校的模板。}

  感谢母校、感谢联盟!

  感谢高德纳!

  感谢提供交大硕士学位论文~\LaTeX~模板的同学!

  开源精神、奉献精神!

  \vskip 20mm
 {\zihao{-3}
  \hspace{80mm} {汪卫}

  \hspace{80mm} {2016年12月}

  \hspace{80mm} {于北京理工大学宇航大楼624} }
\end{thanks}


\end{document}
