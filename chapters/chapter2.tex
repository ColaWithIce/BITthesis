\chapter{如何使用BITthesis模板}
\section{样例项目}
我对研究生院提供的LaTeX 模板进行了较多的修改,使其符合了学院方面的最新格式要求,并且修复了设置字号时行间距不正确等等错误,可以直接使用或者用于参考学习:

\section{稳态波衰减法和脉冲衰减法}
目前已经发展了多种冷流模拟试验,有直接法[198]、稳态波衰减法[199]、频率响应法[200]、脉冲法[7]和阻抗管法[201]。冷气状态与真实发动机工作状态之间的区别是介质温度和成分不同,因而平均声速及特征声阻抗也不同,因此,需要将冷态试验数据做进一步转换才能去表征真实发动机的喷管阻尼系数。通过试验能够测量不同复杂结构的喷管阻尼特性,但是试验方法成本较高而且试验周期较长。近年来,研究人员通过数值方法对喷管阻尼特性进行了计算分析[202][203][204],数值计算结果能够很好地吻合试验结果,为喷管阻尼研究提供了便利的数值工具。
本节将重点介绍稳态波衰减法及脉冲衰减法测量喷管阻尼常数的试验原理,通过借鉴试验原理,提出相应的数值计算方法。采用稳态波衰减法测量喷管阻尼时,首先需要调节声源,使其频率等于模拟燃烧室中的某阶固有频率,待燃烧内建立稳定的驻波以后,突然切断声源,当声源切断后,燃烧室内的压力将以指数形式衰减,通过动态压力监测系统记录测量点的瞬态压力曲线,计算压力衰减系数,计算所得衰减系数即为喷管阻尼,

稳态波衰减法原理是燃烧室空腔内施加一个稳定的、周期性的正弦压力波动信号,待发动机内建立起稳定的波形以后,突然切断稳定的压力波动源,让燃烧室内压力自动衰减,通过记录不同点的瞬态压力曲线,即可得到压力衰减速率。
压力振荡幅值与初始压强幅值之间的关系如下式所示:
\begin{equation}
P=p_0 e^{\alpha t}\label{eq:yaqiang}
\end{equation}
其中,为发动机内的衰减系数。在本研究中,可视为壁面阻尼,气体粘性损失,喷管阻尼等因素的总合,是发动机内的整体阻尼。需要说明的是,由于数值计算仅仅为纯流动仿真,最后所得的衰减系数与理论值会有较大的区别,但是,该方法可用来横向比较不同结构发动机内的阻尼特性,为优化发动机设计提供一定的工程指导。

可利用下式计算阻尼值:
\begin{equation}
\alpha=\dfrac{lnp_2-lnp_1}{t_2-t_1}\label{eq:zuning}
\end{equation}
