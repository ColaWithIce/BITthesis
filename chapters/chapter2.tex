\chapter{如何使用BITthesis模板}
\section{样例项目}
我对研究生院提供的LaTeX 模板进行了较多的修改,使其符合了学院方面的最新格式要求,并且修复了设置字号时行间距不正确等等错误,可以直接使用或者用于参考学习:
出于性能和管理方面的考虑,BITthesis使用分布式的源文件方案,将论文的各个部分(通常以章为单位)分散到tex文件中,然后在主文档main.tex中统一处理。如下展示了一个可能的文件目录情况。
\dirtree{%
		.1 \myfolder{pink}{工作文件夹}.
		.2 \myfolder{cyan}{BITthesis.cls}.
		.2 \myfolder{cyan}{bitthesisextra.cls}.
		.2 \myfolder{cyan}{main.tex}.
		.2 \myfolder{cyan}{chapter}.
		.3 \myfolder{lime}{abstract.tex}.
		.3 \myfolder{lime}{chapter1.tex}.
		.3 \myfolder{lime}{chapterX.tex}.
		.3 \myfolder{lime}{app X.tex}.
    .3 \myfolder{lime}{thanks.tex}.
    .3 \myfolder{lime}{summary.tex}.
		.2 \myfolder{cyan}{figures}.
		.3 \myfolder{lime}{myCat.png}.
		.2 \myfolder{cyan}{reference}.
		.3 \myfolder{lime}{test.bib}.
	}%\dirtree

此处为公式演示:
\begin{equation}
\alpha=\dfrac{lnp_2-lnp_1}{t_2-t_1}\label{eq:zuning}
\end{equation}
\section{构建文档}
xeCJK 是提供 LaTeX 中文支持的宏包,并且依赖于 XeLaTeX,因此,我们需要使用 xelatex 命令进行构建。
LaTeX 在构建交叉索引时需要多次运行,才能最终解析所有的引用,并且期间需要 BibTeX 对参考文献数据库进行处理。因此,一般的手动构建命令是:

1.xelatex main

2.bibtex main

3.xelatex main

4.xelatex main

或者强烈建议采用图形化的编译器(Texstudio)进行编译
