\chapter{\LaTeX{}介绍}
\section{使用\LaTeX{}的写毕业论文的有点}
有人还写过论文,参见\url{http://journals.plos.org/plosone/article?id=10.1371/journal.pone.0115069}
在我看来,最大的优点在于:
\begin{itemize}
	\item 数学公式的自动编号和交叉引用\cite{merry_1975_PenetrationVerticalJets}
	\item 文件干净,随手记事本或者Vim或者nano都能编辑,不像Word的docx解压以后一堆人眼无法阅读的xml文档\cite{mendez_1998_StudyGasVelocity}
	\item 因为文件干净,自动化也很方便,Bash、Python……都可以干活(当然Word也可以通过VBS和C进行很强大的自动化)
	\item 强迫用户以结构化的方式写作,输出的PDF结构树清晰\cite{crawford_1995_FutureLibrariesDreams}
	而Word默认导出PDF是不输出结构的,需要另外勾选,当然如果勾选了的话不比LaTeX差[附图1]
	\item 各种各样的宏包,TikZ这种包估计Word万年都不会有对应的插件\cite{allen_1996_ReviewPerformanceEngineering,bilitza_2014_InternationalReferenceIonosphere,bilitza_2012_MeasurementsIRIModel,bilitza_2008_InternationalReferenceIonosphere,zhanghui_2012_WoGuoCuiHuaLieHuaGongYiJiZhuFaZhanYuQuShi}
	\item 模板质量都很高,各种边距都考虑得很周到,而且切换方便,可以管理的格式很多,如[1]中提到的分栏问题Word的模板是解决不了的,因为本质上Word里“分栏”是页面的属性而不是段落的属性UNIX-friendly
	\item 长度单位不依赖于系统的地区设置各种特殊页面界定清晰,修改灵活,不像Word的“封面”功能有些莫名奇妙~
	\item 矢量图只要用了合适的包和编译引擎就能支持很多格式,不像Word只支持emf或者wmf题注系统比Word强到不知道哪里去了
    \item Computer Modern系列字体是真的美,美出声。
\end{itemize}


\section{公式举例}
附录有跟多类型的公式,可以观摩此公式编辑功能的美:

线性稳定性主要研究小扰动振幅的变化规律,主要基础为燃烧室内的一维波动方程。燃烧室内压力$P$可表示为:
\begin{equation}
P=\dot{p}+p_0 e^{\alpha t} e^{j(\omega t+hx)}\label{eq:pressue p}
\end{equation}
若$\alpha>0$,小扰动有增长趋势,则燃烧不稳定;若$\alpha>0$,则小扰动有减弱趋势,燃烧具有稳定性,其增长常数α可表示为各种增益、阻尼效果之和,如式\ref{eq:zengyi}所示。
\begin{equation}
\alpha=\alpha_{pc}+\alpha_{vc}+\alpha_{dc}+\alpha_n+\alpha_p+\alpha_{mf}+\alpha_{g}+\alpha_{w}+\alpha_{st}\label{eq:zengyi}
\end{equation}

\section{一些题外话}
\subsection{内容为王}
论文当然是内容为王,不应该是被格式分心
\begin{itemize}
	\item 不会用LaTeX --> 无法编译 没有文档
\item	不会用word --> 文档真难看 格式丑死了
	
\item 	会用LaTeX --> 漂亮的文档
\item	会用word --> 文档
	
\item	LaTeX 用的好 --> 牛逼的文档
\item	Word 用的好 --> 牛逼的文档
\end{itemize}

\subsection{word VS \LaTeX{}}
能用LaTeX的人,通常知道如何正确地使用LaTeX;能用Word的人,大多数根本就不会正确地使用Word,比如样式模板、“内容和样式分开管理”、域代码、VBA……而且上面好多人说的LaTeX可以直接套现成的模板……那是模板的功劳,幸好本文编写好了北理工的研究生模板啦!

总之:
\begin{itemize}
	\item word是开始觉得容易,后来觉得难,并且发现越来越难
	\item latex是开始觉得难,后来觉得容易,往后又发现难而且非常难,所以就凑合着用了,好在模板很多
\end{itemize}

\section{本文主要研究内容}
关于此模板的使用


